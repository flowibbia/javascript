\documentclass{article}

\usepackage[a4paper, total={7in, 10in}]{geometry} %for manage the page geometry
\usepackage{amsthm}
\usepackage[utf8]{inputenc}
\usepackage{graphicx}
\usepackage{amssymb}
\usepackage{algorithm}
\usepackage{algorithmic}

\begin{document}

\author{Nico Fiorini}
\date{}

\title{Problema grafo bipartito }
\maketitle
\section{Problema 5.2, grafo bipartito}

Risolvere il seguente problema: \\
Un grafo non orientato G=(V,E) si dice bipartito se è possibile
partizionare i suoi vertici in due sottoinsiemi $V_{1}$ e $V_{2}$ 
tali che non esistono archi tra vertici nello stesso sottoinsieme: 
formalmente, se (u,v) $\in$ e, allora u $\in v_{1}$ e $u \in v_{2}$
o viceversa. 
\begin{enumerate}
  \item  Dimostrare che un grafo è bipartito se e solo se non contiene 
         cicli di lunghezza dispari, dove la lunghezza è il numero di 
         archi nel ciclo.
  \item  Descrivere un algoritmo per verificare se un grafo connesso è bipartito. 
  l’algoritmo deve esibire la partizione $V_{1}$ e $V_{2}$ nel caso di 
  risposta affermativa, e un ciclo di lunghezza dispari nel caso di risposta negativa. 
  \item Discutere il tempo di esecuzione dell'algoritmo proposto.
\end{enumerate}

\section*{Soluzione 1.}

\subsection{Lemma 1} 
Sia G=(V,E) un grafo non orientato, ogni componente connessa di G è bipartita,
allora G è bipartito.

\textbf{Dimostrazione} 

Supponiamo per assurdo che G non sia Bipartito, allora esiste un arco $(x,y)$ 
tale che $x,y \in V_{1}$ o $V_{2}$. 
Ciò è assurdo poichè l'arco deve appartenere ad una componente connessa siccome y
è raggiungibile da x, ciò è assurdo in quanto nega l'ipotesi di partenza $\blacksquare$ 

\subsection*{Teorema 1}
Un grafo è bipartito se e solo se non contiene cicli di lunghezza
dispari.


\textbf{Dimostrazione} \\
Procedo dimostrando un'implicazione per volta. 

\textbf{1° implicazione} $\Rightarrow$ \\ 
Si vuole dimostrare che se un grafo è bipartito $\Rightarrow$ non contiene 
cicli di lunghezza dispari. \\
Supponiamo per assurdo che esiste un ciclo di lunghezza dispari: \\ 
$\exists \hspace{1mm} C = (w_{1},w_{2},. . . . w_{n},w_{1})$ tale che $(w_{i-1},w_{i}) \in E$ $ \forall \hspace{2mm} 1 \leq i \leq n$
e n = 2k+1 dove $k \in \mathbb{Z} $

Senza perdita di generalità sia: \\ 

\begin{center}
$w_{1} \in V_{1}$  \\
$w_{2} \in V_{2}$  \\ 
\end{center}

Sia $w_{i} \in v_{1} \hspace{2mm} \forall i=2k+1  \rightarrow w_{n} \in V_{1}$\\
È assurdo in quanto $(w_{n},w_{1}) \in E$ \hspace{1mm} e  $w_{n},w_{1} \in V_{1}$ 

\textbf{2° implicazione} $\Leftarrow $ \\
È sufficiente mostrare quest'implicazione per un grafo connesso, se il grafo non 
è connesso, basta ripetere la dimostrazione per ogni componente connessa di G in 
virtù del lemma 1.


Sia $G=(V,E)$ un grafo senza cicli di lunghezza dispari\\
Sia $w \in V$, posso partizionare il grafo in due sottoinsiemi: 

\begin{center}
  $V_{1}$ = \{ $ v \in V \: | d(v,w) \:$ è pari  \} \\ 
  $V_{2}$ = \{ $ v \in V \: | d(v,w) \:$ è dispari  \} \\ 
  $V_{1} \cap V_{2} =  \emptyset \;\;\; V_{1} \cup V_{2} = V$
\end{center}

Voglio dimostrare che questa partizione è di un grafo bipartito. \\
Supponiamo per assurdo che questa partizione non lo sia,
quindi deve esistere un arco che collega 2 vertici dello stesso insieme, formalmente :

\begin{center}
  $\exists (a,b) \in E \;$ tale che $a,b \in V_{1} $ oppure $ a,b \in V_{2}$ 
\end{center}

Senza perdita di generalità, sia $a=w \Rightarrow d(a,w) = 0$ quindi $a \in V_{1} $ 
ed anche $b \in V_{1} \rightarrow d(w,b) = d(a,b) = 2k$. \\
È assurdo poichè: 
\[1=d(a,b)= 2k\] \\
Quindi si conclude che $a \neq b \neq w$.

Siano ora $\pi_{wa}$ e $\pi_{wb}$ i due cammini minimi che vanno rispettivamente da $w $ ad $ a $, 
e da $w$ a $b$.  
I due cammini hanno almeno un vertice in comune, in quanto w è comune a tutti e due i cammini, 
tuttavia può avere anche altri vertici in comune.

Sia $x$ l'ultimo vertice in comune nei due cammini, una rappresentazione grafica è riportata 
nell'immagine seguente: 
  
%immagine
\begin{center}
\includegraphics[width=3in,keepaspectratio]{scheme.png}
\end{center}

La lunghezza dei due cammini $\pi_{wa}$ e $\pi_{wb}$ sono tutti e due pari o tutti e due dispari,
poichè $a$ e $b$ appartengono allo stesso insieme, e siccome sono cammini minimi, la lunghezza dei
cammini corrisponde alla distanza tra gli estremi dei cammini. Grazie a questa proprietà possiamo
dire che la lunghezze di $\pi_{w,x}^{'}$ e $\pi_{w,x}^{''}$ sono uguali, quindi le 
lugghezze di  $\pi_{xa}$ e $\pi_{xb}$ continuano ad essere tutte e due pari o tutte e due dispari. 

Calcoliamo la lunghezza del seguente ciclo:

\begin{center}
  $C=(\pi_{ax},\pi_{xb},\pi_{b,a})$ \\
  $l(C)= l(\pi_{ax})+l(\pi_{xb})+l(\pi_{b,a})) = d(a,x)+d(x,b)+d(b,a)$
\end{center}
Sappiamo che $l(\pi_{ax}), l(\pi_{xb})$ sono tutte e due pari o tutte e due dispari, di conseguenza
 $d(a,x)+d(x,b)=2k$ inoltre $d(b,a) = 1$ siccome sono vertici connessi da un arco per ipotesi, quindi:

 \begin{center}
  $l(C)=2k+1$ 
 \end{center}
 
È assurdo poichè in partenza abbiamo supposto che non esistono cicli di lunghezza
dispari, quindi possiamo concludere che il partizionamento non può avere archi
che collegano due vertici dello stesso insieme.
 $\blacksquare$
\\

Ricapitolando, abbiamo supposto che non esistono cicli di lunghezza dispari,
abbiamo partizionato il grafo in due sottoinsiemi,e abbiamo provato
a trovare un arco che unisce due vertici dello stesso insieme.d
Facendo questo arriviamo all'assurdo di non soddisfare l'ipotesi di partenza, quindi
il grafo è bipartito, e può essere caratterizzato dal partizionamento trovato.

\section*{Soluzione 2.} 

Basandoci sul teorema precedente, è possibile realizzare un algoritmo che dato un 
grafo in input, è in grado di riconoscere se è bipartito provando a trovare 
un ciclo di lunghezza dispari nel grafo.

Possiamo ricavare l'algoritmo basandoci su una visita BFS, in quanto l'albero 
della visita BFS gode di alcune proprietà utili per il nostro problema.

\subsection*{Lemma 2}
Siano dati un grafo non orientato e connesso $G=(V,E)$, un vertice $s$ in G, ed 
un albero T prodotto dall'algoritmo visitaBFS. Per ogni vertice $v \in T$, risulta 
$l(v)=d_{sv}$ 
\\

Questa lemma è particolarmente utile per sfruttare il partizionamento usato nella 
dimostrazione del teorema 1, in quanto possiamo marcare i vertici con livello di 
profondità pari dell'alberto BFS nell'insieme $V_{1}$, e marchiamo nell'insieme 
$V_{2}$ i vertici con profondità dispari nell'albero BFS. 

\subsection*{Proprietà 1}
Ogni arco di un grafo non orientato G su cui si effettua la visita in ampiezza 
può essere classificato in tre gruppi rispetto all'albero BFS prodotto:

\begin{enumerate}
  \item archi dell'albero BFS.
  \item archi tra vertici allo stesso livello dell'albero BFS.
  \item archi tra livelli adiacenti dell'albero BFS.
\end{enumerate}

\begin{center}
  \includegraphics[width=3in,keepaspectratio]{bfs_tree.png}
\end{center}

L'arco di tipo 2, è un arco che unisce due vertici della stessa profondità 
dell'albero BFS, quindi unisce due vertici dello stesso insieme, di conseguenza basta 
controllare con una variante dell'algortimo visitaBFS, che non ci siano archi 
di tipo 2, nel momento in cui abbiamo un arco di tipo 2, siamo certi della 
presenza di un ciclo dispari, mentre se l'algoritmo arriva a visitare tutto il grafo
senza incontrare archi di tipo 2, allora l'algoritmo non ha archi che uniscono
due vertici dello stesso insieme.

\begin{algorithm}
%\caption{}
\begin{algorithmic}[1]
\STATE \textbf{bool isBipartite(Grafo G,Node s)}
\vspace{1em}
\STATE Marca tutti i vertici di G come inesplorati
\STATE Marca $s$ come esplorato
\STATE Assegna ad s il livello 0  
\STATE T $\leftarrow$ albero formato dalla radice s
\STATE Coda F
\STATE F.enqueque(s)
\WHILE{not F.isEmpty()} 
\STATE$u \leftarrow F.dequeque()$ 
\STATE visita $u$
\FORALL{$arco(u,v) \in E$}  
    \IF{$v$ è inesplorato}  
      \STATE marca v come esplorato
      \STATE F.enqueque(v)
      \STATE rendi $u$ padre di $v$ in T
      \STATE livello di $v \leftarrow $ (livello di u) +1
    \ENDIF 
    \IF{livello di u == livello di $v$}
      \STATE StampaCicloDispari($u,v$)
      \RETURN false
    \ENDIF
\ENDFOR
\ENDWHILE
\STATE StampaInsiemi(s)
\RETURN true
\end{algorithmic} 
\end{algorithm} 

\newpage

Dati i due vertici uniti da un arco di tipo 2 della proprietà 1, è possibile 
stampare i valori dei nodi contenuti nel ciclo, risalendo di padre in padre 
nei 2 vertici, fino a quando non si incontra il vertice in comune.

\begin{algorithm}
%\caption{}
\begin{algorithmic}[1]
\STATE \textbf{void StampaCicloDispari(Node u,Node v)}
\vspace{1em}

\STATE bool common := false
\STATE Sia X un array ausiliario
\STATE int i:=0
\WHILE{not common} 
\PRINT valore di $u$
\STATE X[i] = valore di $v$
\STATE $i++$
\STATE $u \leftarrow$ parent($u$)
\STATE $v \leftarrow$ parent($v$) 
\IF{valore di $u$ = valore di $v$}
\PRINT valore di u
\STATE common:=true
\ENDIF
\ENDWHILE
\WHILE{(i$>$0)}
\PRINT valore di v 
\STATE $i--$
\ENDWHILE
\end{algorithmic} 
\end{algorithm} 

Nel caso il grafo è bipartito, vogliamo stampare le due partizioni trovate. 
Lo facciamo tramite la funzioni stampaInsiemi(s). La funzione parte 
dal vertice s, ed esegue una visita dei nodi nell'albero BFS, stampando prima i 
nodi che appartengono ad un livello di profondità pari, e memorizziamo in un array 
i nodi che appartengono ad un livello di profondità dispari, in modo da poterli 
stampare dopo aver stampato quelli del primo insieme.

\begin{algorithm}
%\caption{}
\begin{algorithmic}[1]
\STATE \textbf{void StampaInsiemi(Node s)}
\vspace{1em}

\STATE Sia X un array ausiliario
\STATE int i:=0
\STATE Coda C
\STATE C.enqueque(s) 
\PRINT Nodi dell'insieme $V_{1}:$
\WHILE{not C.isEmpty()} 
  \STATE $u \leftarrow$ C.dequeque
  \IF{$u \neq null$}
    \IF{valore u \% 2 ==0}
      \PRINT valore di u
    \ELSE 
      \STATE X[i++] = valore di u
    \ENDIF
  \ENDIF
  \FORALL{$i \in$ figli di $u$}
    \STATE C.enqueque(i)
  \ENDFOR
\ENDWHILE 
\STATE int j=0; 

  \PRINT Nodi dell'insieme $V_{2}$
\WHILE{$j < i$}
\PRINT X[j++]
\ENDWHILE
\end{algorithmic} 
\end{algorithm}

\newpage

\section{Soluzione 3.}

Il costo computazionale dell'algoritmo "isBipartite" trovato per la Soluzione 2
è lo stesso ottenuto per una visita di un grafo, quindi $O(m+n)$ nel caso il grafo è 
implementato con liste di adiacenza o liste di incidenza. 
Questo perchè 
L'algoritmo StampaCicloDispari e StampaInsiemi richiede tempo $O(n)$ in entrambi i 
casi, e viene eseguito solo uno di questi due algoritmi una sola volta durante l'algoritmo isBipartite,
per questo motivo, i due algoritmi non incidono sull'andamento asintotico dell'algoritmo.
\end{document} 